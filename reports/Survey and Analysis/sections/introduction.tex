\chapter{Introduction}

\section{Background}

Fantasy Premier League (FPL) has become a popular pastime for millions of football fans, combining the excitement of live sports with strategic gameplay. Initially introduced for the 2002/03 season, FPL is the most popular Fantasy Football game globally, with over 11 million users as of 2024 \cite{fpl}. Users build teams consisting of real-life players, earning points based on their performances in real matches.

As competition becomes increasingly challenging, effective strategies and data-driven decisions has never been more important. FPL's complexity goes beyond selecting players and lining them up in a team, the game offers users the chance to perform transfers and use chips strategically to increase their chances of winning. Due to this, leveraging machine learning and optimisation techniques could be key to improving performance.

\section{Aims and Objectives}

This project aims to develop a machine learning system capable of performing among the top players in FPL. By leveraging historical data, predictive models and optimization strategies the system aims to optimise team management and overall performance in the game. The ultimate goal is demonstrating that these tools can achieve a competitive advantage.

To achieve this aim, the project will pursue the following objectives:

\subsection{Collect and preprocess historical FPL and football data}

The data collection process includes player performance metrics, fixture schedules and other contextual information required for predictive modelling.

\subsection{Player performance predictor}

The ability to simulate a season is required to fully automate FPL. The overall aim is to leverage regression analysis to predict how each player will perform through each Gameweek in a season. This project will compare different techniques and use careful feature selection to find the optimal model for performance predictions.

\subsection{Team selection}

Given that the FPL dataset contains hundreds of players and many valid formations, the project will look at implementing optimisation algorithms to select the best possible combination. In addition, transfers  between Gameweek, can add more complexity to this challenge.
% Maybe mention injuries?

\subsection{Captain and Chip usage}

FPL offers the chance to boost results further by captaining players or utilising different chips. This project will analyse the most efficient strategies to use these and apply them during automation.

\subsection{Evaluate the performance}

To assess the quality of the results, the project will evaluate the system against different baselines and compare the performance of different strategies to find the optimal.