\chapter*{\Large \center Abstract}

% One or two sentences providing a basic introduction to the field, comprehensible to a scientist in any discipline.  Two to three sentences of more detailed background, comprehensible to scientists in related disciplines.  One sentence clearly stating the general problem being addressed by this particular study.  One sentence summarising the main result (with the words ``here I show'' or their equivalent).  Two or three sentences explaining what the main result reveals in direct comparison to what was thought to be the case previously, or how the main result adds to previous knowledge.  One or two sentences to put the results into a more general context.  Two or three sentences to provide a broader perspective, readily comprehensible to a scientist in any discipline.

Fantasy Premier League (FPL) is a globally popular game that combines strategic gameplay with real-world football performances. The complexity of FPL, which includes player selections, transfers, and strategic chip usage, makes it ideal to benefit from a data-driven approach to obtain a competitive advantage. This can be done by applying advanced predictive and optimisation techniques.

This report explores the development of a machine learning-based system to optimise FPL performance. The project aims to demonstrate the effectiveness of these methods through a multi-stream analysis tailored to player positions. The system's performance is evaluated through a comparison of multiple strategies, including both purely data-driven methods and approaches inspired by top-performing FPL managers, looking to find the optimal approach.

Through this work, the project demonstrates how artificial intelligence and statistical analysis can address challenges in data integration, temporal dependency modelling, and strategy optimization.